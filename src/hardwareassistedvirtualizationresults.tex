\begin{frame}
\frametitle{\sout{Hardware-assisted virtualization}}

\begin{center}
Measurements collected without hardware-assisted virtualization:
\end{center}

\begin{table}[]
\centering
\caption{Chess}
\begin{tabular}{ll}
\hline
Paravirtualization & Software Rasterization \\ \hline
$\approx 1$ sec. & $\approx 7$ sec. \\
$\approx 2$ sec. & $\approx 10$ sec. \\
$\approx 3$ sec. & $\approx 13$ sec. \\ \hline
\end{tabular}
\end{table}

\begin{table}[]
\centering
\caption{Julia}
\begin{tabular}{lll}
\hline
Paravirtualization & Software Rasterization \\ \hline
$<\frac{1}{10}$ sec. & $\approx 1$ min. \\
$<\frac{1}{10}$ sec. & $<2$ min. \\
$\approx \frac{1}{10}$ sec. & $\approx 3$ min. \\ \hline
\end{tabular}
\end{table}

%% \begin{itemize}
%% 	\item Without hardware-assisted virtualization, hit to software rasterization is roughly two orders of magnitude compared to hardware acceleration
%% 	\item Hit to paravirtualization is often less than an order of magnitude
%% 	\item Paravirtualization renders up to three orders of magnitude faster than software rasterization
%%         \item This entails that the Chess benchmark is accelerated by paravirtualization
%%         \item We conclude that effects of paravirtualization increase by one order of magnitude if hardware-assisted virtualization is not available
%%         \item Workloads that are otherwise sub-optimal are thus accelerated
%% \end{itemize}

\end{frame}
