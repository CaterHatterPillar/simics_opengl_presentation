%% mascots.tex

\documentclass[conference]{IEEEtran}
% If IEEE MASCOTS is part of the IEEE Computer Society- range of
% conferences, be sure to add the compsoc option to the document class.
% If IEEEtran.cls has not been installed into the LaTeX system files,
% manually specify the path to it like:
%\documentclass[conference]{../sty/IEEEtran}

\usepackage{hyperref}

% correct bad hyphenation here
\hyphenation{op-tical net-works semi-conduc-tor}

% These should probably be merged into the same bibliography list in
% the end, but it might be good to have them seperately now so that
% it's easier to filter out possibly unreliable sources.
\newcommand{\dvtcmdcitebib}[2][]{\cite{#2}} % TODO: \masccitebib
\newcommand{\dvtcmdciteref}[2][]{\cite{#2}} % TODO: \mascciteref
%\newcommand{\dvtcmdcitefur}[2][]{\citefur{#2}}

\begin{document}

% Placeholder title:
\title{%
  Accelerating Graphics in\\
  the Simics Full-system Simulator}

\author{\IEEEauthorblockN{Eric Nilsson}
\IEEEauthorblockA{Intel Corporation\\
Email: \href{mailto:eric.nilsson@intel.com}{eric.nilsson@intel.com}}}

\maketitle

\begin{abstract}
Virtual platforms provide benefits to developers in terms of a more rapid development cycle since development may begin before next-generation hardware is available.
However, there is a distinct lack of graphics virtualization in industry-grade virtual platforms, leading to performance issues that may reduce the benefits virtual platforms otherwise have over execution on actual hardware.

This paper demonstrates graphics acceleration by the means of paravirtualizing OpenGL~ES in the Wind~River Simics full-system simulator.
We propose a solution for paravirtualized graphics using magic instructions to share memory between target and host systems, and present an implementation utilizing this method.
The study illustrates the benefits and drawbacks of paravirtualized graphics acceleration and presents a performance analysis of strengths and weaknesses compared to software rasterization.
Additionally, benchmarks are devised to stress key aspects in the solution, such as communication latency and computationally intensive applications.

We assess paravirtualization as a viable method to accelerate graphics in system simulators; this reduces frame times up to 34 times compared to that of software rasterization.
Furthermore, magic instructions are identified as the primary bottleneck of communication latency in the implementation.
\end{abstract}


% For peer review papers, you can put extra information on the cover
% page as needed:
% \ifCLASSOPTIONpeerreview
% \begin{center} \bfseries EDICS Category: 3-BBND \end{center}
% \fi
%
% For peerreview papers, this IEEEtran command inserts a page break and
% creates the second title. It will be ignored for other modes.
\IEEEpeerreviewmaketitle

\section{Introduction}
Citation\dvtcmdcitebib{journals:lee:2006}\dvtcmdciteref{web:microsoft:2013:warp}.

\section*{Acknowledgment}
\ldots

\bibliographystyle{IEEEtran}
\bibliography{mascots}

\end{document}

% Notes:
% no keywords
% conference papers do not normally have an appendix
