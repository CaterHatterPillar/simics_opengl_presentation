%% mascots.tex

\documentclass[conference]{IEEEtran}
% If IEEE MASCOTS is part of the IEEE Computer Society- range of
% conferences, be sure to add the compsoc option to the document class.
% If IEEEtran.cls has not been installed into the LaTeX system files,
% manually specify the path to it like:
%\documentclass[conference]{../sty/IEEEtran}

\usepackage{hyperref}

% correct bad hyphenation here
\hyphenation{op-tical net-works semi-conduc-tor}

\begin{document}

% Placeholder title:
\title{%
  Accelerating Graphics in\\
  the Simics Full-system Simulator}

\author{\IEEEauthorblockN{Eric Nilsson}
\IEEEauthorblockA{Intel Corporation\\
Email: \href{mailto:eric.nilsson@intel.com}{eric.nilsson@intel.com}}}

\maketitle

\begin{abstract}
One morning, when Gregor Samsa woke from
troubled dreams, he found himself transformed in his bed into
a horrible vermin. He lay on his armour-like back, and if he
lifted his head a little he could see his brown belly,
slightly domed and divided by arches into stiff sections.

The bedding was hardly able to cover it
and seemed ready to slide off any moment. His many legs,
pitifully thin compared with the size of the rest of him,
waved about helplessly as he looked. "What's happened to me?"
he thought.

It wasn't a dream. His room, a proper human
room although a little too small, lay peacefully between its
four familiar walls.

A collection of textile samples lay spread
out on the table - Samsa was a travelling salesman - and above
it there hung a picture that he had recently cut out of an
illustrated magazine and housed in a nice, gilded frame.

It showed a lady fitted out with a fur
hat and fur boa who sat upright, raising a heavy fur muff that
covered the whole of her lower arm towards the viewer. Gregor
then turned to look out the window at the dull weather.
\end{abstract}

% For peer review papers, you can put extra information on the cover
% page as needed:
% \ifCLASSOPTIONpeerreview
% \begin{center} \bfseries EDICS Category: 3-BBND \end{center}
% \fi
%
% For peerreview papers, this IEEEtran command inserts a page break and
% creates the second title. It will be ignored for other modes.
\IEEEpeerreviewmaketitle

\section{Introduction}
Citation\cite{IEEEexample:article_typical}.

\section*{Acknowledgment}
\ldots

\bibliographystyle{IEEEtran}
\bibliography{mascots}

\end{document}

% Notes:
% no keywords
% conference papers do not normally have an appendix
